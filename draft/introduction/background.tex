\section{Why active matter, matters}
Potential applications, where it's used in the real world, why a theoretical explanation is important

\section{Framework for design of active matter systems}

\subsection{System design of active systems}

\subsection{Particle design of active systems}

\section{Theoretical understanding of active matter}

\subsection{Critical onset}
MIPS, Redner, collision theory

\subsection{Emergent behavior}
Much less here. Dumbbells, binary snowmen.

\section{Universality classes of active matter}
What is a universality class, and why does it matter?
What work has been done on understanding the universality classes of active matter?
Is this work good, or no?

\begin{figure*}[t]
\begin{center}
\includegraphics[width=\textwidth]{../figures/framework.pdf}
\label{fig:design_space}
\caption{
The design space for active matter can be broken into two components: System Design and Particle Design.
In changing aspects of these two design components, we seek to engineer the Critical and Emergent Behavior a system exhibits.
In this work (indicated by blue stars), we vary the stoichiometry (in the creation of a 2-component mixture) and implict interactions (shape), and identify emergent behaviors not previously seen in active matter systems.
Our discovery of novel active behavior by tuning just a narrow set of parameters in this space highlights the richness of this design space, and the opportunity for further theoretical work to understand the role of System and Particle Design on system behavior.
(Seminal citations for each design component are included in the text.)
}
\end{center}
\end{figure}
