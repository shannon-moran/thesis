% For each chapter's introduction:
% What: What is the overall problem we seek to solve?
% Why: What would this tell us more broadly?
% How: What is a  creative way of solving one of those open questions that we'll use in this paper?
% What: We are able to answer one of these narrowly-defined questions with the following answers.
% What: What do we find?
% Why: Why is it important?

% =========
% INTRODUCTION
% =========

% The goal of the introduction is to place this work in the broader scope of human knowledge
% What problems will this help solve? Why would anyone care to read this?
Active matter is a field of rapidly expanding interest and research activity over the last decade \cite{Ramaswamy_2010_AnnRevConMatPhys,Marchetti_2013_RevModPhys,Bechinger_2016_RevModPhys,Marchetti_2016_CurrentOpinionColloidInterfaceScience}.
Vicsek's pioneering work showed a collection of point particles with alignment rules displays rich collective behavior, including phase separation \cite{Vicsek_1995_PRL}.
However, theoretical work describing the collective behavior of bacteria demonstrates that phase separation behavior is not reliant upon explicit alignment rules \cite{Cates_2010_PNAS}.
In a phenomenon known as ``motility-induced phase separation'' (MIPS), systems of disks were found to phase separate as a consequence of density-dependent particle velocity \cite{Cates_2013_EPL}.
This phase separation behavior of isotropic particles has been explained using a variety of models, including: athermal phase separation \cite{Fily_2012_PRL}, the kinetic steady-state balancing of particle fluxes \cite{Redner_2013_PRE, Redner_2013_PRL}, classical nucleation \cite{Richard_2016_SoftMatter,Redner_2016_PRL}, and the balancing of collision and ballistic timescales \cite{Bruss_2018_PRE}.
Importantly, phase separation predicted by theory has been observed in experiments, which confirm the activity-dependent formation of clusters and ``active crystals'' \cite{Palacci_2013_Science,Petroff_2015_PRL,Briand_2016_PRL}.

However, in real-world systems particles (e.g. bacteria) are rarely isotropic in shape.
Thus, one thrust in the active matter community has focused on understanding how particle anisotropy will change the behavior theoretically predicted for systems of isotropic particles.
In a simple anisotropic model, simulations of rods with varying aspect ratios and densities display a rich variety of collective motion, such as laning, swarming, and jamming \cite{Wensink_2012_JPhysConMat,Wensink_2012_PNAS,Yang_2010_PRE}.
Additionally, simply changing the direction of the driving force relative to a fixed particle shape (e.g. ``rough'' triangles) drastically alters the resulting collective behavior and onset of phase separation \cite{Wensink_2014_PRE,Ilse_2016_JChemPhys}.

Few general mechanisms have been proposed for the varying impacts of particle anisotropy on collective behavior.
Active squares display a steady state ``oscillatory'' regime in which large clusters break up and re-form \cite{Prymidis_2016_SoftMatter}.
Mixtures of gear-shaped ``spinners'' with opposite rotational driving forces phase separate through competing steric interactions \cite{Nguyen_2014_PRL, Sabrina_2015_SoftMatter, Spellings_2015_PNAS}.
In systems of active ``dumbbells'', particle anisotropy allows for the stabilization of cluster rotation \cite{Suma_2014_EPL, Cugliandolo_2017_PRL}.
This cluster rotation is also observed in active squares \cite{Prymidis_2016_SoftMatter}, but is notably absent in clusters of frictionless isotropic particles.

From these studies, we can see a general description of the impact of particle shape anisotropy on the phase separation and emergent system behavior is needed.
Such a description would allow us to tailor the form and onset of critical behavior in active systems through ``implicit'' steric means, rather than explicit interaction rules.

In this thesis, I computationally and theoretically describe the role of particle shape on the phase separation and emergent behavior of 2D self-propelled polygons in an active matter system.
This thesis is organized as follows.

In Chapter \ref{ch:background}, I provide an overview of the progress in designing systems of active, self-propelled particles and theoretical descriptions of their critical and emergent behavior motivating this work. In Chapter \ref{ch:methods}, I outline the 2D active polygon model and dynamics I use for Chapters \ref{ch:active-shapes}-\ref{ch:percolation},

In Chapter \ref{ch:active-shapes}, I perform a systematic study of the impact of particle shape on the phase separation behavior in active systems. I find that structure in the phase-separated cluster resembles that of the shape's densest packing in equilibrium systems. I develop a method for quantifying the impact of an effective interaction (e.g. shape) on the onset of phase separation in ``collision efficiency'', capturing the ability of inter-particle collisions to slow down particles during motility-induced phase separation (MIPS). Importantly, this also allows me to explain a previously-observed steady-state ``oscillatory'' regime as a natural consequence of particle shape in active systems.

In Chapter \ref{ch:binary-shapes}, I investigate a simple method of varying inter-particle collisions at a system level: through mixing particle types. I uncover emergent phenomena not yet seen in active Brownian particle systems, including microphase separation and stable fluid clusters that can coexist at steady state with solid clusters and a sparse gas phase. I quantify a measurable, implicit steric attraction between active particles as a result of shape and activity. This provides the first evidence that implicit interactions in active systems, even without explicit attraction, can lead to a calculable effective preferential attraction between particles. Importantly, that the narrow slice of system and particle design space I investigate still exhibits such rich emergent phenomena highlights the potential for a wide variety of behavior to be accessible to active matter systems through simple parameter designs.

In Chapter \ref{ch:percolation}, I take a fundamental approach to understanding the impact of particle shape in active systems by investigating whether systems of different self-propelled shapes fall into different universality classes. By mapping active matter near the order-disorder onto a non-equilibrium percolation model, I identify three distinct cluster evolution curve collapses of similar-behaving shapes. I further find that the quasi-critical behavior of the cluster scaling distribution predicts the same groupings as the collapsed cluster evolution curves. Further work will be needed to confirm the critical exponents, including a more rigorous finite size scaling approach to identify the critical point of each system.

The work in this thesis opens a variety of intriguing questions about the potential for effective interactions to be a rich avenue for engineering active matter systems. I conclude with thoughts on future work in Chapter \ref{ch:outlook}.

This thesis contains work that is currently in preparation or undergoing peer review. As of May 25, 2020 \textcolor{blue}{(Will update with status as of final submission of this thesis)}, the results contained in Chapter \ref{ch:active-shapes} have been submitted for peer review, while the projects reported in Chapters \ref{ch:binary-shapes} and \ref{ch:percolation} are currently in preparation.
