% The goal of the introduction is to place this work in the broader scope of human knowledge
% What problems will this help solve? Why would anyone care to read this?

% What: What field is this in, and what are the important questions?
Active Matter: why it matters.Wat can we do with it?
phase separation even in the absence of explicit interaction
we see all this novel phenomena
can we understand what's causing it? can we develop rules around this?

% What: What are some previous related open questions, and how have those been solved?
In this thesis, I look to quantify the impact of one such interaction on the emergent and critical behavior of active systems.

% Why: What problems will answering these questions help us solve?
critical behavior: could help us understand thermodynamics
emergent behavior: when we look to design behavior of these systems, we need to understand how components of the design space can lead to particular types of behavior


% This thesis is organized as follows (from Aaron's papers)
This thesis is organized as follows.
In Chapter \ref{ch:background}, we provide an overview of the work to date in the design of systems of active, self-propelled particles and theoretical descriptions of their critical and emergent behavior.
In Chapter \ref{ch:methods}, we review the model system and computational techniques that we use to study systems of active self-propelled particles with shape.
In Chapter \ref{ch:active-shapes},
In Chapter \ref{ch:binary-shapes},
In Chapter \ref{ch:percolation},
We conclude with highlighting areas of future study opened up by this thesis in Chapter \ref{ch:conclusions}.

In chapter 2, we provide a theoretical overview of both the thermodynamics and dynamics of liquid-solid transitions. In chapters 3 and 4 we review the model systems and computational techniques that we use to study these transitions. In chapter 5, we study the formation of metastable ordered solids in the con- text of dodecagonal quasicrystals. Quasicrystals have useful photonic properties on the micro/nanoscale [27,28], and serve as a useful case study for our investigations. In chapter 6, we investigate the mechanism underlying the glass transition. To do so, we characterize the dynamics of several model glass formers, as well as an experimental granular system of macroscopic ball bearings that exhibits many of the same dynamical characteristics. In chapter 7, we introduce a new class of computer algorithms for characterizing assembled structures, which can be applied to studying thermodynamic transitions in assembled sys- tems. We conclude by providing possible avenues for future study in chapter 8, and some closing remarks in chapter 9. While the work presented here is largely focused on the fundamental aspects of liquid-solid transitions, we have applied our knowledge of these topics to several novel self-assembly applications, which we briefly review in Appendix A. The simulation and analysis codes that we have developed to study liquid-solid transitions are reviewed in Appendix B.

% For each chapter's introduction:
% What: What is the overall problem we seek to solve?
% Why: What would this tell us more broadly?
% How: What is a  creative way of solving one of those open questions that we'll use in this paper?
% What: We are able to answer one of these narrowly-defined questions with the following answers.
% What: What do we find?
% Why: Why is it important?

This thesis contains work that is currently in preparation or undergoing peer review. As of May 25, 2020 \textcolor{red}{(Will update with status as of final submission of this thesis)}, Chapter \ref{ch:active-shapes} has been submitted for peer review, while Chapters \ref{ch:binary-shapes} and \ref{ch:percolation} are currently in preparation.
