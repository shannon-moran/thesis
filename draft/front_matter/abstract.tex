% Can't be longer than 550 words
% currently at 532

Murmurations of birds, schools of fish, and herds of migrating animals are macroscale examples of self-propelled units exhibiting emergent collective behavior. We call such systems ``active matter''. On the colloidal scale, active matter systems of self-propelled particles exhibit a rich array of emergent phenomena with the potential for useful responsive and transport behaviors. A particularly intriguing aspect of active matter is that it is a non-equilibrium phase transition with no explicit equilibrium mapping, as active systems undergo phase separation from a sparse gas phase to a dense phase-separated regime even in the absence of explicit interparticle attraction. One way of tailoring the interactions between particles without adding explicit interaction terms is to give particles shape, thereby adding steric effective interactions to the system. Understanding the role of effective interactions on the phase separation and emergent behavior in active matter systems will be of great utility as the field looks to engineer targeted behavior through particle and system design.

In this thesis, I computationally and theoretically describe the role of particle shape on the phase separation and emergent behavior of 2D self-propelled polygons in an active matter system.

In the first project, I perform a systematic study of the impact of particle shape on the phase separation behavior in active systems. I find that structure in the phase-separated cluster resembles that of the shape's densest packing in equilibrium systems. I develop a method for quantifying the impact of an effective interaction (e.g. shape) on the onset of phase separation in ``collision efficiency'', capturing the ability of inter-particle collisions to slow down particles during motility-induced phase separation (MIPS). Importantly, this also allows me to explain a previously-observed steady-state ``oscillatory'' regime as a natural consequence of particle shape in active systems.

In the second project, I investigate a simple method of varying inter-particle collisions at a system level: through mixing particle types. I uncover emergent phenomena not yet seen in active Brownian particle systems, including microphase separation and stable fluid clusters that can coexist at steady state with solid clusters and a sparse gas phase. I quantify a measurable, implicit steric attraction between active particles as a result of shape and activity. This provides the first evidence that implicit interactions in active systems, even without explicit attraction, can lead to a calculable effective preferential attraction between particles. Importantly, that the narrow slice of system and particle design space I investigate still exhibits such rich emergent phenomena highlights the potential for a wide variety of behavior to be accessible to active matter systems through simple parameter designs.

In the third project, I take a fundamental approach to understanding the impact of particle shape in active systems by investigating whether systems of different self-propelled shapes fall into different universality classes. By mapping active matter near the order-disorder onto a non-equilibrium percolation model, I identify three distinct cluster evolution curve collapses of similar-behaving shapes. I further find that the quasi-critical behavior of the cluster scaling distribution predicts the same groupings as the collapsed cluster evolution curves. Further work will be needed to confirm the critical exponents, including a more rigorous finite size scaling approach to identify the critical point of each system.

% This thesis opens a variety of intriguing questions about the potential for effective interactions to be a rich avenue for engineering active matter systems. I conclude with thoughts on future work to build upon the foundation I lay here.
