% Can't be longer than 550 words --Jim's was 200ish, very short. Intro sentence with follow-up summaries of each paper.

% what is the field
% what is the promising area or question that, if solved, would help move the field forward?
% what is the thing I clearly demonstrate in this thesis

% project 1
Studies of active particle systems have demonstrated that particle anisotropy can impact the collective behavior of a system, motivating a systematic study.
Here, we report a systematic computational investigation of the role of anisotropy in shape and active force director on the collective behavior of a two-dimensional active colloidal system.
We find that shape and force anisotropy can combine to produce critical densities both lower and higher than those of disks.
We demonstrate that changing particle anisotropy tunes what we define as a  ``collision efficiency'' of inter-particle collisions in leading to motility-induced phase separation (MIPS) of the system.
We use this efficiency to determine the relative critical density across systems.
Additionally, we observe that local structure in phase-separated clusters is the same as the particle's equilibrium densest packing, suggesting a general connection between equilibrium behavior and non-equilibrium cluster structure of self-propelled anisotropic particles.
In engineering applications for active colloidal systems, shape-controlled steric interactions such as those described here may offer a simple route for tailoring emergent behaviors.

% project 2
% motivation + key question to be answered
Studies of self-propelled active particle systems have demonstrated that particle shape can change the emergent and critical behavior of a system through steric-induced changes to particle collisions.
This raises the question: how would multiple steric-induced collision types in a system impact the observed collective behavior?
% how we answered that question in this paper
Here, we computationally explore the emergent behavior found in binary mixtures of self-propelled polygons at different size ratios and stoichiometries.
% what we found
We find emergent phenomena that, to our knowledge, has not been seen in other systems of active matter.
First, we identify microphase separation resulting from ``steric bonding'', a shape-induced and activity-induced preferential attraction we can quantify between like-species in the absence of explicit attractive interactions.
Second, we observe the formation of stable fluid clusters at steady state, resulting from the addition of a first shape type that distrupts the dense-packing formation of the second shape type.
This structural disruption and corresponding increase in particle motility also facilitates the formation of a larger phase-separated regime.
Finally, we highlight that the same mechanism that enables fluid clusters can also stabilize a three-phase steady state with a fluid cluser, solid cluster, and sparse gas phase.
% why this is important
Importantly, all the observed behavior is sensitive to chosen design parameters (size ratio, stoichiometry).
That we find such rich phenomena in investigating just a narrow slice of the design space for self-propelled active particles highlights the potential for a wide range of engineered behavior in such systems.

% =============
% =============
% From Zachary Sherman's thesis abstract-- pretty solid

% A diverse set of functional materials can be fabricated using dispersions of colloids and nanoparticles. If the dispersion is responsive to an external field, like dielectric and charged particles in an electric field or paramagnetic particles in a magnetic field, the field can be used to facilitate self-assembly and control particle transport. One promising feature of field-responsive materials is the ability to drive them out of equilibrium by varying the external field in time. Without the constraints of equilibrium thermodynamics, out-of-equilibrium dispersions display a rich array of self-assembled states with useful material and transport properties. To leverage their unique behaviors in real applications, a predictive, theoretical framework is needed to guide experimental design.
%
% In this thesis, I carry out a systematic investigation of the self-assembly and dynamics of colloidal dispersions in time-varying external fields using computer simulations, equilibrium and nonequilibrium thermodynamics, and electro-/magnetokinetic theory. I first develop efficient computational models for simulating suspensions of polarizable colloids in external fields. The simulations are accurate enough to quantitatively reproduce experiments but fast enough to reach the large length and time scales relevant for self-assembly. I use this simulation method to construct the complete equilibrium phase diagram for polarizable particles in steady external fields and find that many-bodied, mutual polarization has a remarkably strong influence on the nature of the self-assembled states. Correctly accounting for mutual polarization enables a thermodynamic theory to compute the phase diagram that agrees well with simulations and experiments.
%
% Though the equilibrium structures are crystalline, in practice, dispersions typically arrest in kinetically-trapped, disordered or defective metastable states due to strong interparticle forces. This is a key difficulty preventing scalable fabrication of colloidal crystals. I show that cyclically toggling the external field on and off over time leads to growth of colloidal crystals at significantly faster rates and with many fewer defects than for assembly in a steady field. The toggling protocol stabilizes phases that are only metastable in steady fields, including complex, transmutable crystal structures. I use nonequilibrium thermodynamics to predict the out-of-equilibrium states in terms of the toggle parameters. I also investigate the transport properties of dispersions of paramagnetic particles in rotating magnetic fields. Like toggled fields, rotating fields also drive dispersions out of equilibrium, and their dynamics can be tuned with the rotation frequency.
%
% I find that the rotating field greatly increases particle self-diffusivity compared to steady fields. The diffusivity attains a maximum value several times larger than the Stokes- Einstein diffusivity at intermediate rotation frequencies. I develop a simple phenomenological model for magnetophoresis through porous media in rotating fields that predicts enhanced mobility over steady fields, consistent with experiments. Lastly, I study the nonlinear dynamics of polarizable colloids in electrolytes and report a new mode of electrokinetic transport. Above a critical external field strength, an instabilty occurs and particles spontaneously rotate about an axis orthogonal to the field, a phenomenon called Quincke rotation. If the particle is also charged, its electrophoretic motion couples to Quincke rotation and propels the particle orthogonally to the driving field, an electrohydrodynamic analogue to the Magnus effect.
%
% Typically, motion orthogonal to a field requires anisotropy in particle shape, dielectric properties, or boundaries. Here, the electrohydrodynamic Magnus (EHM) effect occurs for bulk, isotropic spheres, with the Quincke rotation instability providing broken symmetry driving orthogonal motion. In alternating-current (AC) fields, electrophoresis is suppressed, but the Magnus velocity persists over many cycles. The Magnus motion is decoupled from the field and acts as a self-propulsion, so I propose the EHM effect in AC fields as a mechanism for generating a new type of active matter. The EHM "swimmers" behave as active Brownian particles, and their long-time dynamics are diffusive, with a field-dependent effective diffusivity that is orders of magnitude larger than the Stokes-Einstein diffusivity. I also develop a continuum electrokinetic theory to describe the electrohydrodynamic Magnus effect that is in good agreement with my simulations.
