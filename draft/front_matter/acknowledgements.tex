It's fitting that acknowledgements are the beginning of a thesis.
Without you all, this PhD process would not have happened-- thank you.

First, thanks goes to my advisor, Professor Sharon Glotzer.
When I visited Michigan when deciding whether to go to grad school, I wasn't sure that I should be going to graduate school at all-- until I met you.
Thank you for setting an example of a creative, innovative, and impactful scientist that I've aspired to emulate.

One of the many things that has made my time at Michigan so special has been the support of the faculty and the collegial academic community.
Professor Robert Ziff, I've so appreciated our chats about science, Ann Arbor, and life over the last few years.
To my thesis committee-- Sharon, Bob, and Professor Xiaoming Mao and Professor Jordan Horowitz-- thank you for taking the time to improve this work before it went into print.
Professor Lola Eniola, I'm so grateful for your mentorship these last few years as we worked together on the Peer Mentorship program. I learned a lot about working in academia, effecting change, and professional advancement from you.

Thanks is also due to mentors who helped get me to this point.
Professor Matt Helgeson, thank you for taking on an enthusiastic undergrad and joyously showing me what it meant to do real science. Our venture to NIST for beam time in a blizzard was the moment I realized that I could be a scientist, and that moment of realization still brings me joy nearly 10 years later.
Dr. Adam Rothman, thank you for supporting me in going back for a doctorate post-BCG. Five years after we last work together, I still think back often on the lessons you taught me about management, passion for work, and my own leadership style.

My time in the Glotzer lab has been spent working on active matter. Thank you to the original ``active matter subgroup'' I had the pleasure of starting my PhD journey with-- Isaac Bruss (a coauthor on the paper behind Chapter \ref{ch:active-shapes}), Matthew Spellings, Chengyu Dai, and Mayank Agrawal.

My time in graduate school has been blessed by outstanding ``co-workers'' I could not have gotten through this time without.
Will Zygmunt, thanks for keeping me laughing when I should have been working, for always being down for a walk and coffee, and for keeping me more grounded than I would have ever been without your friendship. As I've handed in my thesis before you: ``Marco''.
Vyas Ramasubramani and Bradley Dice, how lucky I am to have worked with people whose company I so enjoy and who have been such technical role models for me. I'm a better computational scientist thanks to both of you.
Karen Coulter, among many more impactful things I should list here, thank you for giving me the push I needed to take up hockey. MACHRL was one of the great joys of my time in grad school.
Other colleagues-turned-friends, in alphabetical order-- Rosy Cersonsky Adorf, Simon Adorf, Jim Antonaglia, Julia Dschemuchadse, Chrisy Du, Erin Teich-- thank you keeping me laughing through this PhD process.

Thanks also goes to friends outside of the Glotzer lab who helped me keep an eye on the world outside of academia.
Kavita Chandra, thank you for assuring me that I would eventually finish this thing (and for being a great friend, etc etc).
Alison Banka, thanks for being my sounding board in grad school. I look forward to finding more chip dip recipes in common.

Finally, I would never have taken the step of coming to graduate school without the support of my family.
Mom and Dad, thank you for always enabling me to push myself, for loving me, and showing me what it means to be a good person.
Louie and Colin, being your sister has been one of the greatest gifts of my life. When I think of family, it's our trio that I think of.
Last but not least-- Alice, none of this would have been possible without your unpaid labor as my editor-in-chief. What a gift it is to love and be loved by you.

\begin{center}\rule{5in}{1pt}\end{center}

This thesis is based upon work supported by the National Science Foundation Graduate Research Fellowship under Grant No. DGE 1256260.
I also acknowledge financial support for my graduate studies as an ACM SIGHPC and Intel Computational and Data Science Fellow, and from the Point Foundation as a Point Scholar.

The projects in this work were supported as part of the Center for Bio-Inspired Energy Science, an Energy Frontier Research Center funded by the U.S. Department of Energy, Office of Science, Basic Energy Sciences under Award \# DE-SC0000989.
Work in Chapter \ref{ch:active-shapes} was supported in part through computational resources and services supported by Advanced Research Computing at the University of Michigan, Ann Arbor.
Work in Chapter \ref{ch:active-shapes} and \ref{ch:percolation} used resources of the Oak Ridge Leadership Computing Facility, which is a DOE Office of Science User Facility supported under Contract DE-AC05-00OR22725.Work in Chapters \ref{ch:binary-shapes} and \ref{ch:percolation} used the Extreme Science and Engineering Discovery Environment (XSEDE), which is supported by National Science Foundation grant number ACI-1548562; XSEDE award DMR 140129.\cite{XSEDE}
