\section{Introduction}

In this paper we aim to develop a generalized description of the role of active particle anisotropy through direct comparison to frictionless active disks (i.e. isotropic particles).
We study a family of translationally self-propelled 2D polygons (of side number $3{\leq}n{\leq}8$) with force director anisotropy implemented as shown in Figure \ref{fig:model}.
This choice of shapes systematically extends previous studies on triangles with inertia \cite{Wensink_2014_PRE}, triangles with friction \cite{Ilse_2016_JChemPhys}, and squares \cite{Prymidis_2016_SoftMatter}.
Full simulation parameters and additional details can be found in Section \ref{sec:methods}.

We show that the onset of phase separation at a critical density $\phi^*$ is highly dependent on the shape of the particle given constant $\Pe$.
In our system, we observe phase separation at densities as low as $\phi^*=0.01$ in vertex-forward 6-gons, or as high as $\phi^*=0.37$ in edge-forward 3-gons-- both below and above $\phi^*$ of disks.
Interestingly, we find that the direction of the force director is sufficient for changing the $\phi^*$ for a given shape, but not for changing the relative phase separation onset between different shapes.
Specifically, edge-forward active particles have higher $\phi^*$ than their vertex-forward counterparts.
Additionally, the internal structure of the phase-separated cluster is primarily determined by the particle shape and resembles each shape's equilibrium densest packing.
This resemblence suggests a link between structure and critical density not yet explored in active systems.

In addition to this systematic study, this work's contribution to the study of anisotropic active matter is the introduction of a ``collision efficiency'' measure.
We find that systems with the lowest critical densities are also those that maximize particle deceleration per unit increase in inter-particle collision pressure, $P_\text{coll}$.
That is, some shapes can more efficiently convert particle collisions into decreases in particle velocity, $v$, leading to phase separation.
This allows us to quantitatively attribute changes to $\phi^*$ in systems of shapes versus disks to steric impacts on collisions, and directly shows that we can tune critical behavior of active systems by tuning the nature of the inter-particle collision dynamics.

We note that 3- and 4-gons (the only two previously studied active polygons) behave fundamentally differently from other polygons.
We attribute this to the slip planes present in their densest packings.
As these shapes have been used as model systems for a number of previous studies \cite{Wensink_2014_PRE,Ilse_2016_JChemPhys,Prymidis_2016_SoftMatter}, we show why such results should not be generalized to systems that do not have slip planes.



% ======
% METHODS
% ======
\section{Analysis Methods}
\label{sec:methods}

% =====
% Determining critical density
% =====
\subsection{Phase separation density calculation}
\label{sec:critical-density}

Multiple methods exist in the literature to determine the critical density for phase separation in active systems.
In an active system of squares\cite{Prymidis_2016_SoftMatter}, a system was considered ``clustered'' if the fraction of system particles in the largest cluster was ${\geq}0.2$.
However, we found this method to be ill-suited for our systems, some of which are comprised of many small clusters.
In disks, studies have used local-density histograms about randomly-sampled points of the simulation box \cite{Bruss_2018_PRE} or about each particle \cite{Redner_2013_PRL}.
If the histogram displayed two peaks, the system was considered phase separated.
However, the very low system densities studied here limit the efficacy of the random-sample approach (e.g. at a packing fraction of 0.01, the high-density ``peak'' would be ${\leq}2\%$ of the magnitude of the larger peak).
In dumbbells, studies used both a grid-based and Voronoi-based local density calculation to develop local density histograms, to equal effect \cite{Petrelli_2018_EPJ}.

To determine phase separation even at low densities, we calculated two separate histograms of local densities within a $2.5r_\text{max}$ radius (1) of randomly sampled points ($N=\num{1d5}$) and (2) about each particle ($N=\num{1d4}$).
For each shape, $r_\text{max}$ was calculated as the circumscribing radius about the shape.
We then calculated a position-normalized local density histogram of the system by multiplying the frequencies of local densities in each local density bin by one another.
If the resulting histogram has a high-density peak ${\geq}20\%$ the height of the low-density peak, we consider the system to be phase separated.

The onset of this phase separation is characterized by a critical particle density, $\phi^*$, at which the system transitioned from a homogeneous mixture to coexisting low and high density phases.
We define the critical density, $\phi^*$, as the lowest density at which $>50\%$ of the system replicates phase separate.
In Figure \ref{fig:phase_diagram}, error bars are given as the range of densities which have some replicates exhibiting both homogenous and with others exhibiting phase-separating behavior.


%=====
% Structural order in clusters
%=====
\subsection{Structural order in clusters}
\label{sec:order}

We examine internal cluster structure with two order parameters.
We first calculated the $k$-atic order parameter, i.e. the bond-orientation order parameter for $k$-fold rotational symmetry.
\begin{equation}
\psi_k(i) = \frac{1}{n} \sum_j^n e^{ki\theta_{ij}}
\end{equation}
The parameter $k$ governs the symmetry of the order parameter while the parameter $n$ governs the number of neighbors of particle $i$ to average over.
For calculating bond order, $\theta_{ij}$ is the angle between the vector $r_{ij}$ and $(1,0)$, i.e. the angle of the bond between particle $i$ and particle $j$ with respect to the x-axis.
In other systems, $\psi_k$ has been used to identify hexagonal ($k=6$) order in systems of active disks \cite{Redner_2013_PRL} and ordering on a square lattice ($k=4$) in systems of active squares \cite{Prymidis_2016_SoftMatter}.

The body-orientation order parameter tells us relative \textit{orientations} of local particles,
\begin{equation}
\xi_s(j)=e^{{i}s\theta_j}
\end{equation}
taking into account $s$-fold symmetry, where $\theta_j$ is the angle that rotates particle $j$ from a reference frame into a global coordinate system and $i$ is the imaginary unit.
For particles with even $n$, $s=n$; for particles with odd $n$, we set $s=2n$ to account for anti-parallel packings \cite{Atkinson_2012_PRE}.


% =====
% Collision pressure
% =====
\subsection{Collision pressure calculation}
\label{sec:collision-pressure}

% https://hoomd-blue.readthedocs.io/en/stable/module-hoomd-compute.html
In a 2D system of particles, we used HOOMD (v2.2.1) to calculate the instantaneous (scalar) pressure of the system as $P=(2K + 0.5W)/A$, where $K$ is the total kinetic energy, $W$ is the configurational component of the pressure virial, and $A$ is the area of the box.
We can isolate the pressure due to inter-particle collisions, $W/A=\frac{1}{2A}\sum_i\sum_{j\neq{i}}\bm{F_{ij}}\cdot\bm{r_{ij}}=P-\frac{2K}{A}$.
We further normalize the pressure by the thermal energy as $P_\text{coll}\equiv(W/A)/k_BT$ to facilitate comparison among systems of particles, as $k_BT$ is varied by shape to maintain constant $\Pe=150$.
While pressure in equilibrium systems is typically taken over an ensemble, here we use it as an instantaneous measure of the location in configuration space of the system.
This allows us to view particle trajectories in velocity and configuration space, allowing for the definition of a unique master curve for each system.

To calculate each shape's ``trajectory'' through $v/P_\text{coll}$ space shown in Figure \ref{fig:pressure}, we sampled complete simulation trajectories for simulations below, at, and above the critical density, and calculated a distinct $P_\text{coll}$ and average particle velocity ${\langle}v{\rangle}$ for each time step.
We then binned the $P_\text{coll}$ values into equal-size bins, and calculate an overall average ${\langle}v{\rangle}$ and standard deviation of ${\langle}v{\rangle}$ for each bin.
These averages and standard deviations are normalized by the $v_\text{ballistic}$ calculated for each shape, and are plotted against the average $P_\text{coll}$ value in the corresponding bin.


% ======
% RESULTS AND DISCUSSION - Figures
% ======
\begin{figure}[T]
\begin{center}
\includegraphics[width=8.3cm]{active-shapes/figures/phase_separation_main.pdf}
\caption{
Critical density and collective behavior of active anisotropic systems.
(a) Critical density for systems of each $n$-gon.
We define the critical density, $\phi^*$, as the density at which $>50\%$ of the replicates phase separate.
Lower error bar bounds indicate the minimum system $\phi$ at which at least one replicate phase separated, while the upper error bar bounds indicate the minimum $\phi$ at which all replicates phase separated.
(b) Left 4 columns: Representative steady-state local density snapshots in the critical ($\phi^*$) and phase separated ($>\phi^*$) regimes.
A distinctive feature of phase separation in systems of anisotropic particles is the formation of multiple stable clusters that persist for long time scales.
Right column: Representative snapshots of local structure in a phase-separated cluster for each shape, colored by $\xi_s$ as defined in Section \ref{sec:order}.
}
\label{fig:phase_diagram}
\end{center}
\end{figure}

\begin{figure}[t]
\begin{center}
\includegraphics[width=8.3cm]{active-shapes/figures/structure.pdf}
\caption{
Example of structural evolution of clusters in system of vertex-forward 8-gons at $\phi=0.5$.
a) Left column: Active force director $\hat{n}^A$ exhibits strong polarization at all times, pointing towards the center of the cluster both at the boundary and throughout the cluster.
Center column: Hexatic bond order $\psi_6$ forms quickly and uniformly through clusters. Spatial boundaries in the order parameter are the result of cluster mergers that have not yet annealed.
Right column: Body order $\xi_8$ accounts for particle orientation in the cluster. Strong orientational grains form in the clusters, though they do not span clusters as completely as bond order. Grain boundaries are apparent and do not anneal completely.
b) Legend for orientation maps in (a).
c) Snapshots of bond and body order from regions highlighted in (a).
}
\label{fig:structure}
\end{center}
\end{figure}

% ======
% RESULTS AND DISCUSSION-- make full sections for thesis
% ======

% =====
% Phase separation and critical behavior
% =====
\section{Impact of particle shape on the system density of phase separation onset}

As we increase the number of vertices (i.e. become more ``disk-like''), we expected to see monotonically increasing critical density\cite{Dijkstra_2019_JChemPhys} from high-anisotropy 3-gons towards lower anisotropy 8-gons.

Instead, as shown in Figure \ref{fig:phase_diagram}a, phase-separation behavior does not vary monotonically with $n$.
For shapes of  $n=[3,4]$, we observe a $\phi^*$ near that of disks in this Pe regime, with exact value dependent on the force director.
As we increase $n$ to 5, we see a sharp decrease in $\phi^*$ with continued dependence on the force director.
The lowest critical densities are observed for shapes of $n=6$, above which we observe the expected monotonic increase in $\phi^*$ as $n$ is increased to [7,8].\cite{anisotropy_comp}

We first address the impact of the force director.
We expect $\phi^*$ to depend on the nature of  the active force director because the stability of small cluster depends on the force directors, as suggested in the collision example diagram in Figure \ref{fig:model}.
Specifically, for vertex-forward shapes, the only stable dimer sustains translational motion.
For edge-forward shapes, stable dimers exist that are either stationary and/or can sustain translational motion.
Looking only at the mechanical force balance on configurations of edge- versus vertex-forward particle clusters, we might expect that edge-forward particles would phase separate more easily due to more effective inter-particle slowing.
However, the sustained translational motion of small clusters allows increased inter-cluster collisions in the vertex-forward systems.
It is clear that this increased inter-cluster collision phenomena wins out, with lower $\phi^*$ for vertex forward $n=[3,4,5]$.
Following this logic, the translational speed of a vertex-forward dimer relative to the particle ballistic velocity should  decrease with increasing $n$.
We hypothesize that for $n>6$, this decrease in small cluster translational speed leads to the lack of difference between edge- and vertex-forward $\phi^*$.
Representative small-$N$ clusters are shown for each combination of $n$-gon and force director in the Supplementary Information.

In investigating the structures formed by particles in the phase-separated cluster, we find that without exception the particles have assembled into their densest packing, as shown in the far right column of Figure \ref{fig:phase_diagram}b.
Using this information, we make the following observations.
For 6-gons (the shape with the lowest $\phi^*$), the densest packing has neither void space nor slip planes.
For 5-, 7-, and 8-gons, the densest packing has void space, but no slip planes.
For 3- and 4-gons, the densest packing has no void space, but has slip planes.
This leads us to hypothesize that a system's ability to inhibit particle movement in the cluster (where void space and slip planes play a role) is critical to understanding the critical behavior.

Additionally, the only two shapes in our simulations that exhibit an ``oscillatory'' regime in their phase behavior are 3- and 4-gons (\textcolor{blue}{Video links at end of Chapter; working to figure out way to include them in the final submitted PDF.}).
These shapes are also the only two that have slip planes in their densest packings.
In the literature, other studies have noted oscillation as novel behavior accessed via anisotropy and activity \cite{Prymidis_2016_SoftMatter, Feng_2019_SoftMatter}.
We posit that the oscillatory regime for anisotropic particles is in fact a natural consequence of the preferred steady-state structure of the component particle shapes in these systems.
We will revisit this claim more rigorously in Section \ref{sec:collision_efficiency}.

Our final observation on the critical behavior is that the nature of the phase separation varies significantly based on shape, as shown in Figure \ref{fig:phase_diagram}b.
Beyond the critical regime, we see the formation of many stable clusters at steady state for $n{\geq}5$.
This is in contrast to systems of isotropic disks, where secondary cluster formation is short-lived with phase separation characterized by a single large cluster \cite{Redner_2013_PRE, Redner_2013_PRL}.
The phenomena of multiple phase-separated clusters at steady state is theoretically predicted in bacteria \cite{Cates_2010_PNAS}, but not in other theoretical models focused on isotropic active particle phase separation\cite{Fily_2012_PRL,Redner_2016_PRL}.

% =====
% SEC: Cluster nucleation and growth dynamics
% =====
\begin{figure}[t]
\begin{center}
\includegraphics[width=8.3cm]{active-shapes/figures/coarsening.pdf}
\caption{
Example of phase separation kinetics for vertex-forward 5-gons at three system densities ($\phi>\phi^*$).
The fraction of system particles in a cluster, $N_C/N$, is plotted over the evolution of the simulation.
Particles are considered ``in a cluster'' if their local density is $\geq0.6$.
$N_c/N$ trajectories for all ten replicates for each $\phi$ are shown, though the behavior is so similar that the replicates are only distinguishable for $\phi=0.1$.
Snapshots are colored by local density, colorbar shown.
}
\label{fig:coarsening}
\end{center}
\end{figure}

\begin{figure*}[t]
\begin{center}
\includegraphics[width=17.1cm]{active-shapes/figures/velocity_fields.pdf}
\caption{
Particle displacement fields for simulations at steady state, laid over a map of local densities.
(a) Clusters of disks have no net motion, with particle motion limited to the cluster boundaries and gas phase.
(Shown is a system of disks at $\phi=0.3$).
In contrast, clusters of anisotropic particles display both
(b) net rotational motion (shown for edge-forward 7-gons, $\phi=0.1$) and
(c) net translational motion (shown for vertex-forward 4-gons, $\phi=0.5$).
}
\label{fig:velocity}
\end{center}
\end{figure*}


\begin{figure}[t]
\begin{center}
\includegraphics[width=14cm]{active-shapes/figures/small_clusters_chart.pdf}
\caption{
Representative clusters are shown for small clusters of all combinations of shape and force director studied.
Clusters of shapes in dark grey are those that are commonly observed during cluster formation.
Clusters of shapes in light grey are those that can be observed, but are short-lived.
Shaded \textit{N}-mer and shape intersections indicate combinations where there is no stable cluster observed; while it might be possible to theoretically build a cluster of size \textit{N} for these shapes, we do not see such clusters in practice.
}
\label{fig:SI_clusters}
\end{center}
\end{figure}

\section{Cluster growth and coarsening dynamics}

It remains an open question in the literature as to how shape may affect the kinetics of phase separation, e.g. coarsening and domain growth laws in active systems.
Here, we investigate how particle shape enables the observed phase separation initially into multiple small clusters with coarsening at steady state.

Before phase separating, systems exhibit localized areas of high-density fluctuations, as described in many other theoretical studies of active systems \cite{Cates_2013_EPL, Fily_2012_PRL}.
These localized areas of high density are hexagonally ordered, with the exception of 4-gons, which order on a square lattice.
Following this initial structuring, orientational order develops consistent with the known densest packing of each regular polygon \cite{Atkinson_2012_PRE}.
An example of this phase separation process in vertex-forward 8-gons is shown in Figure \ref{fig:structure}.

This transition from random orientation to close-ordered densest packing is due to the active collision pressure on the clusters.
Studies on active disk cluster nucleation have confirmed that inward-pointing particles at the cluster boundary is a necessary condition for nucleation \cite{Redner_2016_PRL,Richard_2016_SoftMatter}.
Similarly, active polygon clusters possess a net-inward force (Figure \ref{fig:structure}a).
However, unlike in clusters of disks, the rotation of $n$-gons within the cluster is sterically inhibited.
Thus, there exists a sustained inward-facing pressure on the clusters driving the structure to a densest packing.

We observe that the nature of the phase separation dynamics for shapes resembles that of quenched disks\cite{Redner_2013_PRL} for $n{\geq}5$, as seen in Figure \ref{fig:phase_diagram}b.
Multiple small clusters form and are stable at steady state (where steady state is determined by the methods described in Section \ref{sec:methods}).
However, the coarsening behavior between shapes differs.
As seen in Fig. \ref{fig:phase_diagram}, the critical-regime onset phase separation for $n=[5,7,8]$ is characterized by the formation of one (or few) clusters that quickly form and slowly grow, while for $n=6$, cluster nucleation is so favorable that we see the nucleation of many small clusters even in the critical regime.

We demonstrate this coarsening behavior in Figure \ref{fig:coarsening}, where the fraction of the system in a cluster ($N_C/N$) is plotted over time (in units of $\tau$, where $\tau$ is the time for a particle to ballistically travel its own diameter).
At low densities, but even at those above the critical system density for a given shape, we observe rapid nucleation and growth of small clusters, which remain stable at steady state (this behavior is also observed in the low density/activity limit of dumbbells \cite{Suma_2014_EPL}).
At higher densities, the size of the clusters increases the likelihood of another cluster colliding with it and merging to make a larger cluster.

This leads us to another key aspect of anisotropic systems not seen in disks: sustained rotational and translational motion of clusters (Figure \ref{fig:velocity}).
Previous studies on squares found that sustained motion drove the system into an oscillatory behavior \cite{Prymidis_2016_SoftMatter}.
We find that such motion is also critical to the coarsening of clusters of active shapes.
In contrast, clustered disks cannot sustain motion, and quenched systems coarsen through the dissolution of some clusters and growth of others rather than inter-cluster collisions.
The only net motion within clusters of disks is at the boundaries, where a balance of particle fluxes in/out characterizes the steady state configuration \cite{Redner_2013_PRL}.
As a result, the steady state of multiple small clusters in a system of isotropic particles is unfavorable, as clusters in such systems are only stabilized by particles being self-propelled into the cluster.


% =====
% SEC: Collision efficiency
% =====

% =====
% FIGURE: Collision efficiency
% =====
\begin{figure*}[t]
\begin{center}
\includegraphics[width=17.1cm]{active-shapes/figures/collision_efficiency.pdf}
\caption{
(a)
Shown are the average trajectories for $5{\leq}n{\leq}8$ in $v/P_\text{coll}$ space for both edge- and vertex-forward particle simulations.
(Note the inverted axis for velocity.)
The nucleation, growth, and steady state regions are highlighted.
Increasing slope of the growth regime in $v/P_\text{coll}$ corresponds to decreased $\phi^*$, and is predictive for shapes with given force director.
Error bars are the standard deviation, with full calculations detailed in Section \ref{sec:collision-pressure}.
Where error bars are not visible, they are smaller than the data marker.
(b)
Trajectories for 3- and 4-gons are plotted separately.
Here, both shapes collapse onto one master curve.
The master curves for edge- and vertex-forward 3- and 4-gons also collapse onto on another.
Error bars are calculated as in (a).
(c)
Individual trajectories are shown for 5- and 3-gons at the indicated $\phi$.
While velocity decreases monotonically with increasing $P_\text{coll}$ for 5-gons, in 3-gons we observe an ``oscillation'' in which the largest cluster in the system breaks up at $\phi=0.50$.
Pressure and velocity snapshots are taken every 100$\tau$.
}
\label{fig:pressure}
\end{center}
\end{figure*}

\section{Collision efficiency as a method of quantifying the impact of shape on phase separation behavior}
\label{sec:collision_efficiency}

Phase separation due to MIPS is the result of particle slowing as local density increases, with $v(\rho)$ \cite{Cates_2013_EPL}.
Here, we demonstrate a method for quantifying the impact of shape on ${dv}/{d\rho}$.

To build our intuition for this approach: at a particle level, we can describe MIPS as collision-induced slowing.
In a system of frictionless disks, collisions between small numbers of particles are not stable, with clusters of small size ($n_C<10$) generally having a short lifespan ($<\tau$).
(Nucleation in disk systems is facilitated by local polarization of the active force directors leading to a stable nucleation seed \cite{Redner_2016_PRL,Richard_2016_SoftMatter}.)
In contrast, collisions between anisotropic particles can create long-lasting clusters of small $n_C$, ``seeds'', such as those highlighted in Figure \ref{fig:SI_clusters}.
In addition to lifetimes lasting $\gg\tau$, some seeds can sustain translational motion and/or stabilize collisions from external particles.
While these seeds are not a necessary condition for phase separation, they facilitate the process by slowing both constituent seed particles and single particles colliding with the seed, leading to localized areas of high density.

At a system level, we can translate this collision-induced slowing to a ``collision efficiency'' during the nucleation and growth of clusters.
We hypothesize that those systems in which collision work is more efficiently transformed into a decrease in average particle velocity (i.e. greater $-{dv}/{d\rho}$) are also those that are able to phase separate at lower system densities (lower $\phi^*$).
As the system density $\phi$ is a proxy for the number of collisions a particle experiences \cite{Bruss_2018_PRE}, particles with higher collision efficiency need fewer collisions-- and thus lower $\phi$-- to reduce the average particle speed and lead to phase-separation of the system.

To demonstrate this quantitatively, we measure the instantaneous pressure $P_\text{coll}$ due to inter-particle collisions (calculations detailed in Section \ref{sec:collision-pressure}).
In Figure \ref{fig:pressure}a, we plot the trajectories of systems through $v/P_\text{coll}$ space.
We find that each system type ($n$ and force direction) falls onto a well-defined trajectory with short nucleation, long growth, and flat steady-state regions.
The slope of this growth regime, $-{dv}/{d\rho}$, is what we term the ``collision efficiency''.
We observe that relative slopes of the growth regimes correctly predict the relative critical densities of the shapes studied, including the relative critical densities of edge-forward and vertex-forward systems of the same shape.

Notably, 3- and 4-gons require significantly higher collision pressure to reach steady state, as shown in Figure \ref{fig:pressure}b.
These systems fall on the same master curve, suggesting that some feature similarity in the system drives similarity in $v/P_\text{coll}$ space.
Using the concept of collision efficiency, we can now quantitatively demonstrate how the slip planes observed in 3- and 4-gon densest packings lead to the ``oscillatory'' behavior discussed earlier and observed in previous works \cite{Prymidis_2016_SoftMatter}.
As shown in Figure \ref{fig:pressure}c, systems of shapes whose densest packings do not have slip planes (like the edge-forward 5-gons shown) proceed monotonically through $v/P_\text{coll}$ space with $\tau$, eventually resulting in phase separation.
In contrast, systems with slip planes do \textbf{not} proceed through $v/P_\text{coll}$ space monotonically with $\tau$.
In the system shown of vertex-forward 3-gons, a phase-separating system proceeds through $v/P_\text{coll}$ space as the phase-separated clusters form.
At high $P_\text{coll}$, however, the system is no longer able to sustain the inter-particle collision pressure and the cluster breaks apart, retracing its path through $v/P_\text{coll}$.
Additionally, the lack of hysteresis in this path through $v/P_\text{coll}$ space during cluster dissolation confirms that this oscillatory phenomenon is not path dependent or a function of simulation protocol, but rather a function of the particle anisotropy alone.
The oscillatory regime can be described as a system's inability to stabilize the inter-particle collision pressure.

In collision efficiency, we have introduced a metric that quantitatively explains how shape impacts the  critical density in active systems.
This framework tells us that we can tune the critical behavior of a system by altering how efficiently particles decelerate other particles in collisions.


% ======
% CONCLUSIONS AND OUTLOOK
% ======
\section{Conclusions}

In this work, we investigated the critical phase behavior of a 2D active matter system of anisotropic particles in which anisotropy was implemented through polygon shape and active force director.
We demonstrated that we can quantitatively describe the critical behavior as a function of ``collision efficiency'', which can be tuned by engineering particle interactions (here, we explore only shape).
Further, we observe that this critical behavior is related to the structure of the component particle shapes' densest packing at equilibrium.

We showed that increasing the efficiency of inter-particle collisions in slowing particles down during cluster growth is a key driver of decreasing critical densities.
This observation is closely related to a number of theoretical developments in the field of active matter.
We can think of this efficiency as a determinable scaling coefficient on the change in particle velocity with local density (${dv}/{d\rho}$) in MIPS \cite{Cates_2010_PNAS}.
Similarly, an analytical determination of the average collision time for an inter-particle collision would allow prediction of critical onset through the balancing of $\tau_\text{collision}$ and $\tau_\text{ballistic}$ timescales \cite{Bruss_2018_PRE}.
Such an analytical determination would need to account for all possible angles of collision between anisotropic particles and all iterations of force anisotropy.

An analytical description linking driving force and anisotropy to collision time may enable prediction of critical system densities.
Additionally, while the nature of the densest packing in equilibrium can be used to explain the structure seen in dense phase-separated regions, further work is needed to elucidate the link between equilibrium packing and non-equilibrium assembly.
As an understanding of the thermodynamics of active matter continues to develop, establishing the phase behavior of active assemblies will be of intense interest as a means of achieving directed, non-equilibrium self-assembly.

While anisotropic active particles are in the early stages of being synthesized in labs they are ubiquitous in nature.
Biology presents us with a number of intriguing test cases for our framework.
How does changing shape (as some biological systems are able to do) impact the $v/P_\text{coll}$ curve?
For systems with explicit attractive interactions, e.g. chemotaxis, how can we formulate that interaction as a collision efficiency?

Finally, while our work reveals a mechanism for how particle anisotropy in 2D drives different collective behavior from that seen in disks, our explanation is quantitatively descriptive but not yet analytically predictive.
Developing a comprehensive predictive theory of how particle anisotropy will impact the critical density would be of great interest to the field.

\section{Videos}

\textcolor{blue}{5/27: Currently confirming how best to include these videos in the main text.}

Videos are publicly accessible at the following link: \\
\href{https://umich.box.com/s/e2z2mbyaueonzww2d1thi03lyadhwypr}{https://umich.box.com/s/e2z2mbyaueonzww2d1thi03lyadhwypr}.

\begin{enumerate}
  \item \verb|n3_VF_phi50.mp4|: Vertex-forward 3-gons at $\phi$=0.50 \\
  This video shows the same simulation system whose trajectory is shown in the $v/P_{coll}$ space in Figure 6 in the main text. The system phase separates as the inter-particle collision pressure ($P_{coll}$) increases. At approximately 3000$\tau$, the system is destabilized as inter-particle pressure builds along slip planes and is released, leading to the break-up of the cluster. This process has been referred to as “oscillation” in other works. Onset of destabilization is seen at 0:14 in the video.
  \item \verb|n4_VF_phi50.mp4|: Vertex-forward 4-gons at $\phi$=0.50 \\
  Clear slip planes are visible in this system that allow the clusters to shear grains off large clusters.
  \item \verb|n5_VF_phi50.mp4|: Vertex-forward 5-gons at $\phi$=0.50 \\
  This system exhibits rapid cluster formation. This system notably phase-separates with a cluster that exhibits the anti-parallel densest packing form common to 5- and 7-gons. See Figure 3 in the main text for a visual of this structure.
  \item \verb|n6_VF_phi50.mp4|: Vertex-forward 6-gons at $\phi$=0.50 \\
  This system also exhibits rapid cluster formation into many stable, small clusters.
\end{enumerate}
